\chapter{Projektmanagement}
\section{Vorarbeiten}
Diese Bachelorarbeit basiert nicht auf einer Projekt2-Arbeit. Die ganze Arbeit entsteht also während einem Semester.

\section{Aufgabenstellung}
\label{sec:aufgabenstellung-messbar}
TODO:
konkrete Aufgabenstellung, was soll raus kommen -> Howto für Studenten der \acrshort{hftm}

\section{Rahmenbedingungen}
Die Rahmenbedingungen dieses Projektes wurden gemeinsam mit dem Betreuer definiert:
\begin{itemize}
	\item Source-Code für \acrshort{lidar}-Ansteuerung von \acrshort{hftm}, verwenden als Library
	\item Library \verb|ch.quantasy.mqtt.gateway|\cite{ch.quantasy.mqtt.gateway} von Reto Koenig als Schnittstelle zwischen Hardware, User und Event-Bus
\end{itemize}

\section{Projektplan und Meilensteine}

\section{Meetings}
Grundsätzlich mit Dozent alle zwei Wochen
\subsection{Protokolle}
\subsubsection{Startvorlesung mit Herr Anrig (Mo., 19.02.2018)}
\begin{itemize}
	\item Aufgabenstellung erhalten
	\item Termine allgemein, alles auf Moodle
	\item Video ab diesem Semester Pflicht
	\item Eintrag im Book, Termin folgt
	\item Plakat Vorlagen etc., Termin folgt
\end{itemize}

\subsubsection{Treffen mit Reto (Fr., 23.02.2018)}
\begin{itemize}
	\item Code für Java-GUI Software erhalten (Zip, Mail)
	\item Pushed auf https://gitlab.ti.bfh.ch/kilcm1/hftm-lidar
	\item Einführung in die Github Repos von knr1 (Reto) -> https://github.com/knr1?tab=repositories
\end{itemize}
Ziele: Lernbuch für HFTM (entweder in Doku integriert oder extern)


\subsubsection{Treffen mit Reto (Do., 08.03.2018)}
Erste Schritte mit dem GatewayClient gezeigt (LidarService). Weitere Schritte:
\begin{itemize}
	\item Servant und kleiner UI-Service (console writer)
	\item danach dokumentieren
	\item expert termin vorschlag (Dr. Joachim Wolfgang Kaltz)
\end{itemize}
\subsubsection{Treffen mit Experte (Mo., 19.03.2018)}
\begin{itemize}
	\item Vorhaben erklärt, HFTM und Studenten erläutert
	\item Experte ist einverstanden: die Arbeit könne gut ein Handbuch für die Studenten der HFTM sein. Kein Problem.
	\item Referenz-Setup mit den zwei Services und dem Servant gezeigt (Lidar, Console Client, Servant)
\end{itemize}
Weitere Schritte: Weiteres Meeting mit Experte, Alain und Reto Mitte Mai.
\subsubsection{Treffen mit Reto (Do., 22.03.2018)}
nächste Schritte: Doku
\subsubsection{Treffen mit Reto und Alain (Do., 05.04.2018)}
abgesagt da Reto Krank -> verschoben auf Fr., 13.04.2018, 09:00
\subsubsection{Treffen mit Reto und Alain (Do., 13.04.2018)}
\begin{itemize}
	\item Sourcecode vom Robi erhalten
	\item Niveau ist nicht sehr hoch, darf die Studenten nicht gleich überfordern. Java wird in 34 Lektionen vor dem Arbeiten am Robocup erst ausgebildet
	\item Robi hat Intel i5-Basierter PC drauf (also doch kein Raspi) = ebenbürdig mit meinem Lenovo Yoga 2 Pro
\end{itemize}
\subsubsection{Treffen mit Reto (Do., 19.04.2018)}
abgesagt (Reto krank)

\section{Journal}
TODO: Bild vom Excel einfügen am Schluss