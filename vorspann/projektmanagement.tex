\chapter{Projektmanagement}
\section{Vorarbeiten}
Diese Bachelorarbeit basiert nicht auf einer Projekt2-Arbeit. Die Bachelorarbeit entsteht also während einem Semester.

\section{Aufgabenstellung}
\label{sec:aufgabenstellung-messbar}
TODO:
konkrete Aufgabenstellung, was soll raus kommen -> Howto für Studenten der \acrshort{hftm}

\section{Rahmenbedingungen}
Die Rahmenbedingungen dieses Projektes wurden gemeinsam mit dem Betreuer definiert:
\begin{itemize}
	\item Source-Code für \acrshort{lidar}-Ansteuerung von \acrshort{hftm}, verwenden als Library
	\item Library \verb|ch.quantasy.mqtt.gateway|\cite{ch.quantasy.mqtt.gateway} von Reto Koenig als Schnittstelle zwischen Hardware, User und Event-Bus
\end{itemize}

\section{Projektplan und Meilensteine}
TODO
\subsection{fixe Termine}
\begin{table}[H]
	\centering
	\begin{tabular}{lll} \toprule
		\textbf{Datum} 			& \textbf{Termin Beschreibung}		& \textbf{Bemerkungen}		\\ \midrule
		Fr., 01.06.2018, 12:00		& Poster für Ausstellung 		& E-Mail an Heinz.Kipfer@bfh.ch	\\ \midrule
		Mo., 04.06.2018 		& Book - Freigabe Inhalte		& https://students.book.bfh.ch	\\ \midrule
		Mo., 11.06.2018 		& Book - Freigabe Layout		& https://students.book.bfh.ch	\\ \midrule
		Do., 14.06.2018, 20:00		& Abgabe Bericht			&  				\\ \midrule
		Fr., 15.06.2018, 10:40 - 10:55	& Präsentation Arbeit (Finaltag)	&  				\\ \midrule
		Fr., 22.06.2018			& Abgabe Film				& hochladen in Moodle		\\ \midrule
		Mi., 27.06.2018, 10:00 - 12:00	& Verteidigung				&				\\ \bottomrule
	\end{tabular}
	\caption{fixe Termine während der Bachelorarbeit}
	\label{tab:fixeTermine}
\end{table}


\section{Meetings}
Mit dem Betreuer wurden zu Beginn dieser Arbeit die Meetings auf jede zweite Woche festgelegt. Bei Bedarf können Meetings auch Adhoc stattfinden.
\subsection{Protokolle}
\subsubsection{Startvorlesung mit Herr Anrig (Mo., 19.02.2018)}
\begin{itemize}
	\item Aufgabenstellung erhalten
	\item Termine allgemein, alles auf Moodle
	\item Video ab diesem Semester Pflicht
	\item Eintrag im Book, Termin folgt
	\item Plakat Vorlagen etc., Termin folgt
\end{itemize}

\subsubsection{Treffen mit Reto (Fr., 23.02.2018)}
\begin{itemize}
	\item Code für Java-GUI Software erhalten (Zip, Mail) -> Pushed auf https://gitlab.ti.bfh.ch/kilcm1/hftm-lidar
	\item Einführung in die Github Repos von knr1 (Reto Koenig) -> https://github.com/knr1?tab=repositories
	\item Festlegung der Treffen: alle zwei Wochen in Biel
\end{itemize}
Ziele: Lernbuch für HFTM (entweder in Doku integriert oder extern)


\subsubsection{Treffen mit Reto (Do., 08.03.2018)}
Erste Schritte mit dem GatewayClient gezeigt (LidarService). Weitere Schritte:
\begin{itemize}
	\item Servant und kleiner UI-Service (console writer) implementieren
	\item Danach Dokumentieren der ersten Erkenntnisse
	\item Vorschlag für Treffen mit Experten (Dr. Joachim Wolfgang Kaltz)
\end{itemize}
\subsubsection{Treffen mit Experte (Mo., 19.03.2018)}
\begin{itemize}
	\item Vorhaben erklärt, HFTM und Studenten erläutert
	\item Experte ist einverstanden: die Arbeit könne gut ein Handbuch für die Studenten der HFTM sein. Kein Problem.
	\item Referenz-Setup mit den zwei Services und dem Servant gezeigt (Lidar, Console Client, Servant)
\end{itemize}
Weitere Schritte: Weiteres Meeting mit Experte, Alain und Reto Mitte Mai.
\subsubsection{Treffen mit Reto (Do., 22.03.2018)}
nächste Schritte: Doku

\subsubsection{Treffen mit Reto und Alain (Do., 13.04.2018)}
\begin{itemize}
	\item Zugriff auf aktuellen Sourcecode vom Roboter erhalten (auf gitlab.com \cite{gitlab.com/solidus/hefei})
	\item Niveau ist nicht sehr hoch, darf die Studenten nicht gleich überfordern. Java wird in 34 Lektionen vor dem Arbeiten am Robocup erst ausgebildet. Vorher haben die Meisten noch nie objektorientiert programmiert.
	\item Robi hat Intel i5-Basierter PC drauf (also doch kein Raspi) = ebenbürdig mit meinem Lenovo Yoga 2 Pro
\end{itemize}

\subsubsection{Treffen mit Reto (Do., 26.04.2018)}
\begin{itemize}
	\item Alternativen mit "Kein Gluelayer"/Servant innerhalb einer Domain ansprechen, aber weiter wie bisher: zwischen jedem Service gibt es den Servant.
	\item Wenn es im Servant zu Ressourcen-Problemen kommt, verwenden wir eine andere Struktur als YAML/JSON (Bsp. Google's Protocol buffers)
	\item nächste Schritte: Doku, Linien-Finder anschauen, Testing
\end{itemize}

\subsubsection{Treffen mit Reto (Do., 03.05.2018)}
\begin{itemize}
	\item Idee Services über Reference zusammenhängen, ohne die Daten durch den Servant zu schleusen, wenn kompatibel. Der Servant würde den entsprechenden Services das mitteilen.
	\item Zerreissen der "Library" für EdgeDetection, da Refactoring notwendig (Bsp. Naming der Koordinatensysteme Polar und Kartesisch)
\end{itemize}

\subsubsection{Treffen mit Reto (Do., 17.05.2018)}
\begin{itemize}
	\item Architekturbild (Datenfluss löschen und beim anderen Architekturbild bleiben)
	\item erklären weshalb nur der letzte LidarIntent genommen wird
	\item zwei Dokument-"Parts": transition und setup (erklären weshalb)
	\item pushen vom EdgeDetector - Stand heute
	\item Adapter auf Service ändern
	\item Annotations einführen für Contract
	\item Umbenennung: Lidar-Service heisst TIM55x und EdgeDetector heisst 2DEdgeDetector
\end{itemize}



\subsection{abgesagte Meetings}
\subsubsection{Treffen mit Reto und Alain (Do., 05.04.2018)}
abgesagt da Reto krank -> verschoben auf Fr., 13.04.2018, 09:00

\subsubsection{Treffen mit Reto (Do., 19.04.2018)}
abgesagt da Reto krank -> verschoben auf Do., 26.04.2018, 15:30

\section{Journal}
TODO: Bild vom Excel einfügen am Schluss