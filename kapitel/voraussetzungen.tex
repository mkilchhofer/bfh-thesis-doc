\chapter{Voraussetzungen}
\label{chap:voraussetzungen}
Dieses Kapitel soll oberflächlich einige Dinge erklären, die für diese Arbeit vorausgesetzt werden.
\section{Maven Projekt}
\label{chap:maven}
Maven ist ein Tool um den Build-Prozess von Software zu unterstützen. Es bietet beispielsweise Unterstützung fürs Managen der Abhängigkeiten, fürs Testing und zur Paketierung.

Die zu entwickelnden Services aus Kapitel \ref{chap:beispielimplementation} verwenden Maven vor allem für das Dependency-Management. Konkret ist Maven in diesem Projekt für folgendes zuständig:
\begin{itemize}
	\item
	Libraries werden nicht erneut als binäres Jar-File ins Git/\acrshort{vcs}\footnote{Version Control System} hochgeladen. Der Sourcecode einer Library soll nur im \acrshort{vcs} der jeweiligen Entwickler sein.
	\item
	Versionen der Libraries werden Zentral definiert. Ein Upgrade einer Library auf eine neue Version lässt sich somit an einem Ort erledigen.
	\item
	Paketierung der Services/Applikationen als Jar-File.
\end{itemize}
Um Maven zu verwendet erstellt man in der verwendeten Entwicklungsumgebung (\acrshort{ide}) ein Maven Projekt. Zentral in einem Maven-Projekt ist die Datei \verb|pom.xml| (\acrshort{pom} steht für \acrlong{pom}). Sie definiert wie oben erläutert die Umgebung für das Java-Projekt.

Dependencies/Libraries sucht man beispielsweise auf \verb|mvnrepository.com|. Alle Libaries in diesem Repository kennt Maven automatisch und kann diese herunterladen. Dafür fügt man im \verb|pom.xml| als Beispiel für ,,log4j'' folgendes ein:
\begin{lstlisting}[language=XML]
<dependencies>
	<dependency>
		<groupId>org.apache.logging.log4j</groupId>
		<artifactId>log4j-core</artifactId>
		<version>2.11.0</version>
	</dependency>
</dependencies>
\end{lstlisting}
Maven übernimmt dann über die \acrshort{ide} den Rest und lädt das Jar und nötigenfalls die Sources an die erfoderlichen Orte herunter.

TODO: Maven kurz erklären, libs gehören nicht als *.jar ins Git
\section{Tests / Unit Tests}
TODO
\section{Git}
TODO: weshalb, was gehört eingecheckt, was nicht
\section{MQTT}
\label{sec:mqtt}
TODO - Stichworte: \acrshort{mqtt}, Broker, Client, publish, subscribe
