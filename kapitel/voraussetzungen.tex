\chapter{Voraussetzungen}
Die \acrshort{hftm} stellte dem Betreuer in einem eigenen Git-Repository den Sourcecode bereit. Diesen habe ich anschliessend als Zip-Datei via E-Mail erhalten. Während dieser Arbeit (am 16. April) habe ich lesenden Zugriff auf das Git-Repository der \acrshort{hftm} erhalten\cite{gitlab.com/solidus/hefei}. Dabei ist aufgefallen, dass beide Code-Repos Projekt-Files der \acrshort{ide} und weitere Daten, die gegen den Home-Pfad des Commiters zeigen, beinhaltet. Auch sind Log-Dateien und Abhängigkeiten (jar-Dateien) im Git des Projekts eingecheckt:
\begin{lstlisting}[caption={Listing der Daten im Git-Repository 'gitlab.com/solidus/hefei'},language=Bash, columns=fixed]
$ du -sch .[!.]* *
17M	.git
4.0K	.gitignore
4.0K	build.xml
24K	config
17M	lib
107M	logs
4.0K	manifest.mf
100K	nbCodeConvJava.zip
152K	nbproject
4.0K	README.md
3.1M	src
143M	total
\end{lstlisting}
Dieses Kapitel soll oberflächlich einige Dinge erklären, die für diese Arbeit vorausgesetzt werden.
\section{Maven Projekt}
Maven ist ein Tool um den Build-Prozess von Software zu unterstützen. Es bietet beispielsweise Unterstützung fürs Managen der Abhängigkeiten, fürs Testing und zur Paketierung. Die zu entwickelnden Services aus Kapitel \ref{chap:beispielimplementation} verwenden Maven vor allem für das Dependency-Management.

Maven ist in diesem Projekt für folgendes zuständig:
\begin{itemize}
	\item
	Libraries werden nicht erneut als binäres Jar-File ins Git/\acrshort{vcs}\footnote{Version Control System} hochgeladen. Der Sourcecode einer Library soll nur im \acrshort{vcs} der jeweiligen Entwickler sein.
	\item
	Versionen der Libraries werden Zentral definiert. Ein Upgrade einer Library auf eine neue Version lässt sich somit an einem Ort erledigen.
	\item
	Paketierung der Services/Applikationen als Jar-File.
\end{itemize}
Um Maven zu verwendet erstellt man in der verwendeten Entwicklungsumgebung (\acrshort{ide}) ein Maven Projekt. Zentral in einem Maven-Projekt ist die Datei \verb|pom.xml| (\acrshort{pom} steht für \acrlong{pom}). Sie definiert wie oben erläutert die Umgebung für das Java-Projekt.

Dependencies/Libraries sucht man beispielsweise auf \verb|mvnrepository.com|. Alle Libaries in diesem Repository kennt Maven automatisch und kann diese herunterladen. Dafür fügt man im \verb|pom.xml| als Beispiel für ,,log4j'' folgendes ein:
\begin{lstlisting}[language=XML]
<dependencies>
	<dependency>
		<groupId>org.apache.logging.log4j</groupId>
		<artifactId>log4j-core</artifactId>
		<version>2.11.0</version>
	</dependency>
</dependencies>
\end{lstlisting}
Maven übernimmt dann über die \acrshort{ide} den Rest und lädt das Jar und nötigenfalls die Sources an die erfoderlichen Orte herunter.

TODO: Maven kurz erklären, libs gehören nicht als *.jar ins Git
\section{Tests / Unit Tests}
TODO
\section{Git}
TODO: weshalb, was gehört eingecheckt, was nicht
\section{MQTT}
\label{sec:mqtt}
TODO - Stichworte: \acrshort{mqtt}, Broker, Client, publish, subscribe
