\chapter{Schlusswort}
Mit dem vorliegenden Dokument ist es möglich, den Software-Entwicklungsprozess an der \acrshort{hftm} zu optimieren. Davon profitieren  insbesondere auch die Studenten, denn durch die neue Architektur wird es zukünftig für sie einfacher, neue Funktionen für den Roboter zu implementieren. Sie müssen die Funktion nicht in die bereits vorhandene Lösung integrieren. Dadurch dass Microservices nur eine Aufgabe oder Funktion erfüllen, kann sich der Entwickler - also der Student - voll und ganz auf diesen konzentrieren. Der Service und somit die Funktion wird erstmals als Ganzes testbar. Das Risiko Fehler in das Gesamtkonstrukt unbewusst einzubauen minimiert sich bei einer Entwicklung in einem dedizierten Service erheblich. Das Design mit vielen kleinen Services kommt meiner Ansicht nach dem häufig wechselnden Team sehr entgegen. Falls sich trotzdem einmal ein Service unerwartet fehleranfällig und als schlecht wartbar herausstellt, könnte dieser von einem nächsten Studenten relativ einfach komplett abgelöst werden durch eine Neuentwicklung dieses einen Services.

Die definierten Muss-Kriterien sind vollumfänglich erfüllt und es konnten sogar einige Kann-Kriterien umgesetzt werden.

Durch das Einarbeiten in die Entwicklungs-Toolchain mit Maven, Continuous Integration, Integration Testing etc. konnte ich erstmals Prozesse der Entwicklungsabteilung meines Arbeitgebers kennenlernen und verstehen. Dies wird sich in Zukunft positiv auf die Zusammenarbeit mit den Entwicklern auswirken.