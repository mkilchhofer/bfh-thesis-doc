\documentclass[11pt,english,german]{report}

% Package import, Document Settings
\usepackage[a4paper,inner=3.5cm,outer=2.5cm]{geometry}
\usepackage[english,ngerman]{babel}
\usepackage[utf8]{inputenc}

% Packages
\usepackage[hyphens]{url}
\usepackage{caption}
\usepackage{latexsym}
\usepackage[T1]{fontenc}
\usepackage{graphicx}
\usepackage{hyperref}
\usepackage{tabularx}
\usepackage{etoolbox}
\usepackage{fancyhdr}
\usepackage{amsthm}
\usepackage{mathtools}
\usepackage[toc, acronym]{glossaries}
\usepackage{lastpage}
\usepackage{float}
\usepackage{makecell}
\usepackage{ltablex}
\keepXColumns
\usepackage{listings}
\usepackage{csquotes}
\usepackage{subcaption}
\usepackage{glossaries}
\usepackage{pdfpages}
\usepackage{fancyhdr}

\usepackage[
backend=biber,
style=alphabetic,
sorting=ynt
]{biblatex}
\addbibresource{datenbanken/references.bib}

\usepackage{titlesec}
\assignpagestyle{\chapter}{fancy}

\pagestyle{fancy}
\fancyhf{}
\rhead{\thepage}
\lhead{LIDAR Einbindung für den Robocup, \the\year}

\renewcommand\labelitemi{--}
\newcommand{\source}[1]{\caption*{Quelle: {#1}} }
\renewcommand{\chaptermark}[1]{\markboth{\MakeUppercase{#1}}{}}

\hypersetup{pdfborder = 0 0 0}


\definecolor{mygreen}{rgb}{0,0.6,0}
\definecolor{mygray}{rgb}{0.5,0.5,0.5}
\definecolor{mymauve}{rgb}{0.58,0,0.82}

\lstset{
	language=Java,
	numbers=left,
	columns=fullflexible,
	aboveskip=5pt,
	belowskip=10pt,
	basicstyle=\small\ttfamily,
	backgroundcolor=\color{black!5},
	commentstyle=\color{darkgreen},
	keywordstyle=\color{blue},
	stringstyle=\color{gray},
	showspaces=false,
	showstringspaces=false,
	showtabs=false,
	xleftmargin=16pt,
	xrightmargin=0pt,
	framesep=5pt,
	framerule=3pt,
	frame=leftline,
	rulecolor=\color{green},
	tabsize=2,
	breaklines=true,
	breakatwhitespace=true,
	prebreak={\mbox{$\hookleftarrow$}}
}

% Glossary
\makeglossaries

\setcounter{secnumdepth}{2}
\setcounter{tocdepth}{1}

\begin{document}
\newcommand{\titel}		{LIDAR Einbindung für den Robocup}
\newcommand{\versionnumber}	{<tbd>}
\newcommand{\versiondate}	{\today}
\newcommand{\datum}		{\today}
\newcommand{\studiengang}	{Informatik, berufsbegleitend}
\newcommand{\autor}		{Marco Kilchhofer}
\newcommand{\betreuer}		{Dr. Reto Koenig, \acrshort{bfh}}
\newcommand{\auftraggeber}	{Alain Rohr, \acrshort{hftm} / Dr. Reto Koenig, \acrshort{bfh}}
\newcommand{\experte}		{Dr. Joachim Wolfgang Kaltz}


% Titelseite
\begin{titlepage}
\pagenumbering{Alph}
\begin{center}

\includegraphics[width=0.08\textwidth]{img/logo/bfh_logo.png}

Berner Fachhochschule | Haute école spécialisée bernoise | Bern University of Applied Sciences
\vspace{20mm}

\newcommand{\HRule}{\rule{\linewidth}{0.3mm}}
\HRule \\[0.4cm]
{\huge \titel}\\[0.3cm]
{\huge \bfseries  }
\HRule \\[2cm]


% Author und Dozent
\vfill
\end{center}
\begin{tabular}{ll}
	Studiengang:  & \studiengang 	\\
	Autor:        & \autor 		\\
	Betreuer:     & \betreuer	\\
	Auftraggeber: & \auftraggeber	\\
	Experte:      & \experte	\\
	Datum:        & \datum		\\
\end{tabular}

\end{titlepage}

\pagenumbering{Roman}

% Vorspann
\chapter*{Management Summary}
Die Höhere Fachschule für Technik Mittelland (HFTM) nimmt seit einigen Jahren mit dem Team ,,Solidus'' am internationalen RoboCup in der ,,Logistics League'' mit Erfolg teil. In der ,,Logistics League'' geht es darum, mit selber entwickelten Roboter in einer Produktionsanlage (Smart Factory) die Zwischenschritte - also die Logistik - zu automatisieren. Der Roboter bekommt beispielsweise den Auftrag aus dem Lager zur Anlage 1 Material zu bringen, oder von Anlage 2 die produzierten Teile abzuholen und ins Lager oder zu einer weiteren Produktionsanlage zu bringen. Die Roboter-Software ist mittlerweile relativ ausgereift für den aktuellen Cup. Das Team besteht jedes Jahr aus neuen Studenten die keinen vertieften Hintergrund in der Softwareentwicklung haben. Sie übernehmen die bestehende Code-Base vom vorherigen Team und müssen diese an die neuen Gegebenheiten anpassen und erweitern. Die historisch gewachsene, monolithische Architektur der Software wirkt diesen Anforderungen aber entgegen.

Das Ziel ist es, eine Softwarearchitektur und ein Entwicklungsvorgehen zu entwerfen, welche die Anforderungen an kurze Einarbeitungszeit und Flexibilität erfüllt. Fordert die geringere Effizienz der Umsetzung zwar mehr Ressourcen auf der Zielplattform, steigert sie dafür aber die Effizienz bei der Entwicklung.

In einer zeitgemässen Tool-Chain für die Entwicklung und einer Architektur basierend auf Microservices wurde die Anbindung eines Sensors (LiDAR) realisiert. Die einzelnen Schritte wurden so dokumentiert, dass sie in Schulungsunterlagen Verwendung finden können. So kann sichergestellt werden, dass zukünftige Entwicklungen voneinander abgekapselt in eigenständigen Microservices stattfinden können und somit keine direkte Abhängigkeit untereinander aufweisen. Die Kopplung und Datenkonversion der einzelnen Services wird mit Hilfe von domänenspezifischen Verbindern (Servants) realisiert.

Die exemplarische Entwicklung aus der Umsetzung zeigt, wie man künftig an der HFTM die Optimierung und Weiterentwicklung der Software angehen könnte. Das resultierende Dokument zu dieser Arbeit wurde so erfasst, dass es den zukünftigen HFTM-Studenten als Leitfaden dienen kann. Die Ausgangslage und die Umsetzung erlauben einen fliessenden Übergang in die neue Architektur.

% Inhaltsverzeichnis
\tableofcontents

% Content
\chapter{Einleitung}
\pagenumbering{arabic}
\section{Aufgabenstellung}
Seit mehreren Jahren bestreitet die \acrshort{hftm} unter der Leitung von Alain Rohr mit dem Solidus Team erfolgreich die internationalen Meisterschaften des RoboCups in der 'Logistics League'.
Das Team kennt sich meisterlich mit der Ansteuerung der Hardware aus, bittet aber die \acrshort{bfh} um Mithilfe beim Software-Engineering.
Die Aufgabe dieser Arbeit ist es, ein Software-Design für die einzelnen Komponenten des verwendeten Roboters zu entwerfen. Angelehnt an die Vorgehensweise 'Domain Driven Design' soll konkret anhand des \acrshort{lidar}s gezeigt werden, wie das erarbeitete Software-Design implementiert und genutzt werden soll. Folgende Merkmale soll das Design mindestens aufweisen:

\begin{itemize}
	\item
	Die Schnittstelle der einzelnen Domänen muss Programmiersprachen-agnostisch sein
	
	\item
	Die Domänen müssen abgekapselt und unabhängig entwickelt und getestet werden können
	
	\item
	Das Design muss klar vorsehen, dass jedes Jahr das Entwicklungsteam komplett ausgetauscht wird
\end{itemize}
Bei dieser Arbeit gilt es auch zu beachten, dass die Software-Fähigkeiten des jeweiligen Entwicklungsteams erst noch ausgebildet werden müssen, es also nötig ist, die Schnittstellen so leicht und verständlich wie möglich zu halten, um nicht eine zu steile Lernkurve als Voraussetzung zu erschaffen.


\bigskip
TODO:
konkrete Aufgabenstellung, was soll raus kommen -> Howto für Studenten der \acrshort{hftm}

\section{HFTM Robocup}
TODO:
Roboter erklären, Schule erklären, Studienrichtung?

\section{Lidar}
TODO:
was ist \acrshort{lidar}, wozu wird es verwendet, Funktionsweise High-Level



\section{Vorarbeiten}
Diese Bachelorarbeit basiert nicht auf einer Projekt2-Arbeit. Die ganze Arbeit entsteht also während einem Semester.

\section{Rahmenbedingungen}
Die Rahmenbedingungen dieses Projektes wurden gemeinsam mit dem Betreuer definiert:
\begin{itemize}
\item Source-Code für \acrshort{lidar}-Ansteuerung von \acrshort{hftm}, verwenden als Library
\item Library ch.quantasy.mqtt.gateway\cite{ch.quantasy.mqtt.gateway} von Reto Koenig als Schnittstelle zwischen Hardware, User und Event-Bus
\end{itemize}

\chapter{Voraussetzungen}
Die \acrshort{hftm} stellte dem Betreuer in einem eigenen Git-Repository den Sourcecode bereit. Dieser habe ich anschliessend als Zip-Datei erhalten. Während dieser Arbeit (am 16. April) habe ich lesenden Zugriff auf das Git-Repository der \acrshort{hftm} erhalten\cite{gitlab.com/solidus/hefei}. Dabei ist aufgefallen, dass beide Repos Projekt-Files der \acrshort{ide} und weitere Daten, die gegen den Home-Pfad des Commiters zeigen, beinhaltet. Auch sind Log-Dateien und Abhängigkeiten (jar-Dateien) im Git des Projekts eingecheckt:
\begin{lstlisting}[caption={Listing der Daten im Git-Repository 'gitlab.com/solidus/hefei'},language=Bash, columns=fixed]
$ du -sch .[!.]* *
17M	.git
4.0K	.gitignore
4.0K	build.xml
24K	config
17M	lib
107M	logs
4.0K	manifest.mf
100K	nbCodeConvJava.zip
152K	nbproject
4.0K	README.md
3.1M	src
143M	total
\end{lstlisting}
Dieses Kapitel soll oberflächlich einige Dinge erklären, die für diese Arbeit vorausgesetzt werden.
\section{Maven Projekt}
TODO: Maven kurz erklären, libs gehören nicht als *.jar ins Git
\section{Tests / Unit Tests}
TODO
\section{Git}
TODO: weshalb, was gehört eingecheckt, was nicht
\section{MQTT}
\label{sec:mqtt}
TODO - Stichworte: \acrshort{mqtt}, Broker, Client, publish, subscribe

\chapter{Software Design}
In diesem Kapitel wird ein mögliches Design für den Robocup erläutert.
\section{Einführung}
Bis anhin hat die \acrshort{hftm} mit dem Team Solidus am Robocup durchaus erfolgreich mitgespielt. So fragt man sich vielleicht, wieso diese Arbeit überhaupt zu Stande kommt. Bis heute entwickelt man mit einem monolithischen Ansatz. Vieles ist eng miteinander verzahnt und die Software lässt sich schwer und mit viel Zeit warten. Der Roboter für den Wettbewerb wird gleichzeitig aber ständig komplexer und es müssen immer wieder neue Dinge (Bsp. Sensoren) integriert werden. Auch müssen sich die Schüler jedes Jahr durchgehend mit allen Schichten (von Hardware bis Business-Logik) neu befassen. In den nachfolgenden Abschnitten soll gezeigt werden, wie man einzelne Funktionen in Domänen unterteilt und dazwischen simple und nachhaltigere Schnittstellen (\acrshort{api}) realisieren kann.



\bigskip
TODO
Stichworte: Microservice vs Monolith, gemeinsamer Event-Bus

\section{Architektur}
Im Kapitel \ref{sec:mqtt} wird beschrieben, dass auf dem Roboter in der Vergangenheit bereits einiges über \acrshort{mqtt} lief. Dies bildet die zentrale Komponente in der neuen Architektur. Es gäbe auch andere Event-Busse, die dafür verwendet werden könnte (Bsp. \acrshort{ros}). \acrshort{mqtt} ist aber bereits weit verbreitet und somit gibt es bereits Libraries und eine Community dafür, aber vor allem kennen die Studenten der HFTM dies bereits. Es sollen Micro-Services entstehen, die nur genau eine Aufgabe erfüllen. Nämlich das hardware-spezifische Protokoll zwischen Hardware (hier \acrshort{lidar}) und der \acrshort{mqtt}-Welt zu adaptieren. \Glspl{service} kennen sich untereinander nicht, sie wissen nichts von der Existenz anderer \glspl{service}. Damit die einzelnen \Glspl{service} miteinander kommunizieren können, kommen so genannte \Glspl{servant} zum Einsatz. Diese kennen die Schnittstellen-"\gls{contract}" (\acrshort{api}) der zu bedienenden Services. Wir werden dieses Konzept in den nachfolgenden Kapitel am Beispiel des \acrshort{lidar} anschauen.

\begin{figure}[H]
	\centering
	\includegraphics[width=0.6\textwidth]{img/architecture/highlevel.png}
	\caption{Architektur in Domains}
	\label{fig:architecture_highlevel}
\end{figure}


\section{Aufbau eines Services, Begriffe}
Jeder Service im System soll sich an in diesem Abschnitt beschriebenen Grundsatz halten. Die Basis dazu liefert das ab dem sechsten Semester kennengelernte Muster mit Intent,Status und Event für die MQTT Topics. Die Library ch.quantasy.mqtt.gateway\cite{ch.quantasy.mqtt.gateway} von Reto Koenig implementiert dies bereits so. Die \acrshort{mqtt}-Topics stellen das \acrshort{api} für den jeweiligen Service dar und liefert bzw. konsumiert eine Dokumentstruktur mit \acrshort{yaml}-Inhalt. Die nachfolgenden Teilabschnitte basieren demzufolge auf dem englischen README von Github\cite{ch.quantasy.mqtt.gateway}. In Abbildung \ref{fig:gatewayclient} ist ersichtlich welche Basis-Topics bzw. \acrshort{api}s in Richtung MQTT-Broker zur Verfügung stellt.

\begin{figure}[H]
	\centering
	\includegraphics[width=0.6\textwidth]{img/ch.quantasy.mqtt.gateway/MqttGatewayClient.png}
    \caption{GatewayClient, ch.quantasy.mqtt.gateway\cite{ch.quantasy.mqtt.gateway}}
    \label{fig:gatewayclient}
\end{figure}



\subsection{Unit}
Als \gls{unit} wir eine Instanz eines Services oder generell eines GatewayClients bezeichnet. Im Hardware-Service für den \acrshort{lidar}-Sensor wird pro konfigurierter Sensor eine Instanz des GatewayClients erstellt. Das sind also unterschiedliche Units.
\subsection{Intent}
Kann ein Service Befehle über die \acrshort{api} auf der Seite von \acrshort{mqtt} entgegennehmen, so werden diese über das Topic '\gls{intent}' dem Service mitgeteilt. Pro Service gibt es nur ein Intent-Topic.
\subsection{Status}
Mehrheitlich statische Informationen stellt ein Service unter dem Basis-Topic '\gls{status}' zur Verfügung. Unter diesem Topic kann eine beliebige Baumstruktur aufgebaut werden.
\subsection{Event}
TODO
\subsection{Description}
Im Topic '\gls{description}' beschreibt ein Service bzw. eine \acrshort{unit} beim Aufstarten bzw. beim instantiieren des GatewayClient das angebotene \acrshort{api}. Das ermöglicht für Entwickler die Anbindung an andere Komponenten, ohne auf deren Seite die Programmiersprache Java mit der Library ch.quantasy.mqtt.gateway\cite{ch.quantasy.mqtt.gateway} zu verwenden.

\section{API Java Servicelogik}

TODO
Beispiels-Listing:
\lstinputlisting[language=Java]{listings/scanner-listener.java}

\chapter{Ausblick}
\section{API / Bindings in seprartem Projekt}
\label{sec:separatebindings}
Beim TiM55x-Service habe ich unbewusst die HFTM-\acrshort{lidar}-Library bis zum Contract durchdrücken lassen. Dies habe ich exemplarisch so beibehalten. Das Problem ist folgendes: der Servant, der mit diesen Service kommuniziert, braucht alle Objekte vom Contract. Die Dependencies eines Services werden beim standardmässigen bauen eines Jars aber nicht eingepackt. Für die Ausführung des Services mit dem Befehl \ctexttt{java -jar paketname.jar} packen wir aber sowieso alle Dependecies in ein Jar, damit lediglich diese Datei aufs Zielsystem deployed / kopiert werden muss. Der Grössenunterschied ist allerdings immens und deshlab sollte ein Servant nicht dieses so genannte Fat-Jar als Depedency reinziehen.

TODO

\begin{itemize}
	\item
	GatewayClient und API an MQTT trennen
	
	\item
	MQTT API versionieren -> 'alte Dependencies' funktionieren weiterhin mit dem 'alten Service'
\end{itemize}

\section{Fehlerbehandlung innerhalb der Services}
TODO

\section{Servant abkürzen}
TODO
Während intensiven Diskussionen mit Reto zum Thema wie sich der Datenaustausch von Service zu Service abkürzen lässt, ist folgende Idee zu Stande gekommen:

\section{Performance Serialisieren / Deserialisieren}
TODO


% Anhang
\chapter{Anhang}
\section{Transitions-Phase Stand Heute zu Microservices}
\label{chap:tranition-git}
Die \acrshort{hftm} stellte dem Betreuer in einem eigenen Git-Repository den Sourcecode bereit. Diesen habe ich anschliessend als Zip-Datei via E-Mail erhalten. Während dieser Arbeit (am 16. April) habe ich lesenden Zugriff auf das Git-Repository der \acrshort{hftm} erhalten\cite{gitlab.com/solidus/hefei}. 

\subsection{Ist-Zustand}
Beim Betrachten des eingecheckten Codes ist aufgefallen, dass beide Code-Repos Projekt-Files der \acrshort{ide} beinhalten. Teils zeigen bei solchen Daten Pfade gegen das User-Home des Commiters. Auch sind Log-Dateien und Libraries/Abhängigkeiten (lib / Jar-Files) im Git des Projekts eingecheckt (siehe Listing \ref{lst:gitlab-listing}). 
\begin{lstlisting}[caption={Listing der Daten im Git-Repository 'gitlab.com/solidus/hefei'},language=Bash, columns=fixed,label={lst:gitlab-listing}]
$ du -sch .[!.]* *
17M	.git
4.0K	.gitignore
4.0K	build.xml
24K	config
17M	lib
107M	logs
4.0K	manifest.mf
100K	nbCodeConvJava.zip
152K	nbproject
4.0K	README.md
3.1M	src
-------------
143M	total
\end{lstlisting}
\subsection{Soll-Zustand}
Auf den \verb|lib|-Ordner kann komplett verzichtet werden, sobald Dependencies über Maven verwaltet werden (siehe Kapitel \ref{chap:voraussetzungen} und \ref{chap:maven}). Libraries werden nicht erneut als binäres Jar-File ins Git/\acrshort{vcs}\footnote{Version Control System} hochgeladen. Der Sourcecode einer Library bleibt ebenfalls nur im \acrshort{vcs} der jeweiligen Entwickler.

Die Ordner \path{logs} und \path{nbproject} können einfach in der \path{.gitignore}-Datei auf die Blacklist gesetzt werden. Logs gehören grundsätzlich nicht ins \acrshort{vcs} und die \acrshort{ide}-spezifischen werden über Maven oder die Entwicklungsumgebung selbst beim Öffnen des Projekts automatisch erstellt. Das Projekt kann vollständig in Maven definiert werden (bsp. die Main-Klassen).

\section{Korrespondenz}
\section{Anderes}



% Abbildungsverzeichnis
\listoffigures

% Glossar ausgeben
% Glossar
\newglossaryentry{contract}{
	name={Contract},
	description={Die Vereinbarung / der Vertrag der Schnittstelle zwischen Services und Servants.},
	plural={contracts}
}
\newglossaryentry{description}{
	name={Description},
	description={Description-Topic eines Services, für API-Dokumentation},
}
\newglossaryentry{event}{
	name={Event},
	description={Event-Topic eines Services, beispielsweise für Messwerte},
	plural={Events}
}
\newglossaryentry{intent}{
	name={Intent},
	description={Befehls-Topic eines Services, zum Absetzen von Steuerungsbefehlen an den Service},
	plural={Intents}
}
\newglossaryentry{payload}{
	name={Payload},
	description={Payload bezeichnet in MQTT die eigentliche Nachricht, also der Inhalt},
	plural={Payload}
}
\newglossaryentry{servant}{
	name={Servant},
	description={},
	plural={Servants}
}
\newglossaryentry{service}{
	name={Service},
	description={},
	plural={Services}
}
\newglossaryentry{status}{
	name={Status},
	description={Status-Topic eines Services, beispielsweise für Zustände und statische Informationen},
	plural={Stati}
}
\newglossaryentry{topic}{
	name={Topic},
	description={Topic bezeichnet in MQTT den Pfad unter welchem eine Nachricht veröffentlicht (published) wird},
	plural={Topics}
}
\newglossaryentry{unit}{
	name={Unit},
	description={eine Instanz eines Services},
	plural={Units}
}

\printglossary[title=Glossar]

% Abkuerzungen ausgeben
% Abkürzungen
\newacronym{api}   {API}   {Application Programming Interface, Programmierschnittstelle}
\newacronym{bfh}   {BFH}   {Berner Fachhochschule}
\newacronym{hftm}  {HFTM}  {Höhere Fachschule für Technik Mittelland}
\newacronym{lidar} {Lidar} {light detection and ranging}
\newacronym{mqtt}  {MQTT}  {Message Queuing Telemetry Transport}
\newacronym{yaml}  {YAML}  {YAML Ain’t Markup Language}
\newacronym{pom}   {POM}   {Project Object Model}
\newacronym{ros}   {ROS}   {Robot Operating System}
\newacronym{loc}   {LOC}   {Lines of Code}
\newacronym{ide}   {IDE}   {Integrierte Entwicklungsumgebung, von englisch: integrated development environment}
\newacronym{vcs}   {VCS}   {Version Control System (Versionsverwaltungssystem)}

\printglossary[type=\acronymtype, title=Abkürzungsverzeichnis]

% Literatur
%\printbibliography
\printbibliography[heading=bibintoc]

%\newpage
%\includepdf[pages={1}]{Selbstständigkeitserklärung.pdf}
\end{document}