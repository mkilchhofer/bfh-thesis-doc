\chapter{Anhang}
\section{Transitions-Phase Stand Heute zu Microservices}
\label{chap:tranition-git}
Die \acrshort{hftm} stellte dem Betreuer in einem eigenen Git-Repository den Sourcecode bereit. Diesen habe ich anschliessend als Zip-Datei via E-Mail erhalten. Während dieser Arbeit (am 16. April) habe ich lesenden Zugriff auf das Git-Repository der \acrshort{hftm} erhalten\cite{gitlab.com/solidus/hefei}. 

\subsection{Ist-Zustand}
Beim Betrachten des eingecheckten Codes ist aufgefallen, dass beide Code-Repos Projekt-Files der \acrshort{ide} beinhalten. Teils zeigen bei solchen Daten Pfade gegen das User-Home des Commiters. Auch sind Log-Dateien und Libraries/Abhängigkeiten (lib / Jar-Files) im Git des Projekts eingecheckt (siehe Listing \ref{lst:gitlab-listing}). 
\begin{lstlisting}[caption={Listing der Daten im Git-Repository 'gitlab.com/solidus/hefei'},language=Bash, columns=fixed,label={lst:gitlab-listing}]
$ du -sch .[!.]* *
17M	.git
4.0K	.gitignore
4.0K	build.xml
24K	config
17M	lib
107M	logs
4.0K	manifest.mf
100K	nbCodeConvJava.zip
152K	nbproject
4.0K	README.md
3.1M	src
-------------
143M	total
\end{lstlisting}
\subsection{Soll-Zustand}
Auf den \verb|lib|-Ordner kann komplett verzichtet werden, sobald Dependencies über Maven verwaltet werden (siehe Kapitel \ref{chap:voraussetzungen} und \ref{chap:maven}). Libraries werden nicht erneut als binäres Jar-File ins Git/\acrshort{vcs}\footnote{Version Control System} hochgeladen. Der Sourcecode einer Library bleibt ebenfalls nur im \acrshort{vcs} der jeweiligen Entwickler.

Die Ordner \path{logs} und \path{nbproject} können einfach in der \path{.gitignore}-Datei auf die Blacklist gesetzt werden. Logs gehören grundsätzlich nicht ins \acrshort{vcs} und die \acrshort{ide}-spezifischen werden über Maven oder die Entwicklungsumgebung selbst beim Öffnen des Projekts automatisch erstellt. Das Projekt kann vollständig in Maven definiert werden (bsp. die Main-Klassen).

\section{Korrespondenz}
\section{Anderes}

