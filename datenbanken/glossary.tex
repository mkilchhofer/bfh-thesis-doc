% Glossar
\newglossaryentry{contract}{
	name={Contract},
	description={Die Vereinbarung / der Vertrag der Schnittstelle zwischen Services und Servants.},
	plural={contracts}
}
\newglossaryentry{description}{
	name={Description},
	description={Description-Topic eines Services, für API-Dokumentation},
}
\newglossaryentry{event}{
	name={Event},
	description={Event-Topic eines Services, beispielsweise für Messwerte},
	plural={Events}
}
\newglossaryentry{gatewayclient}{
	name={Gateway Client},
	description={Verwendete Library für die Kommunikation mit dem MQTT-Broker}
}
\newglossaryentry{intent}{
	name={Intent},
	description={Befehls-Topic eines Services, zum Absetzen von Steuerungsbefehlen an den Service},
	plural={Intents}
}
\newglossaryentry{intellij}{
	name={IntelliJ IDEA},
	description={Verwendete Entwicklungsumgebung (IDE) während der Entwicklung. Version IntelliJ IDEA 2018.1.2 (Ultimate Edition)},
}
\newglossaryentry{payload}{
	name={Payload},
	description={Payload bezeichnet in MQTT die eigentliche Nachricht, also der Inhalt},
	plural={Payload}
}
\newglossaryentry{servant}{
	name={Servant},
	description={Verbinder zwischen zwei oder mehr Services},
	plural={Servants}
}
\newglossaryentry{service}{
	name={Service},
	description={Die Software-Komponente, die eine spezifische Aufgabe erfüllen muss (Bsp. mit dem Sensor kommunizieren)},
	plural={Services}
}
\newglossaryentry{service-logic}{
	name={Service Logic},
	description={Die Logik im Service um eine spezifische Funktion (Service Source) an den GatewayClient anzubinden}
}
\newglossaryentry{service-source}{
	name={Service Source},
	description={Die eigentliche Funktion in einem Service (Bsp. die Berechnung ausführen, das hardwarespezifische Protokoll zu einem Sensor implementieren). Häufig eine Library eines Herstellers zu einem Produkt}
}
\newglossaryentry{status}{
	name={Status},
	description={Status-Topic eines Services, beispielsweise für Zustände und statische Informationen},
	plural={Stati}
}
\newglossaryentry{topic}{
	name={Topic},
	description={Topic bezeichnet in MQTT den Pfad unter welchem eine Nachricht veröffentlicht (published) wird},
	plural={Topics}
}
\newglossaryentry{unit}{
	name={Unit},
	description={eine Instanz eines Services},
	plural={Units}
}
